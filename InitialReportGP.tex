\documentclass[11pt]{article}

\author{KeepItClean}
\title{Traffic Simulator}
\date{February 2016}


\begin{document}


\maketitle
\newpage

\section{Project Description}
\subsection{Project Aim}
Our Traffic Simulator 
\subsection{Project Approach}
	\paragraph{Management Approach: Scrum} The team decided that due the nature of the project it was needed an strategy that allowed progress every predetermine period of time and, at the same time, to have something ready to deliver. For this reason the decision was to follow an Agile development approach using Scrum in order to have a management appropriate for a small development time and  have a specific artifact ready to go at the end of each iteration.
	\paragraph{}The approach we are going to follow is the Scrum methodology focused in the team goals and needs. This means that even when people will be assigned to an specific role, it does not mean that she or he will be only do tasks related. The tasks will be balanced and assigned according to the Sprint number, task complexity, and task completion percentage. It also means that some features of the Scrum development methodology will be modified accordingly and only task progress will be reported.
	\paragraph{}Taking in count the available development weeks, the Sprint length was defined to last two weeks. At the beginning of each sprint a Scrum Planning Meeting will be held. This is where the team will decide what will be the next target deliverable for the sprint. The daily stand up meeting will be replaced by mechanisms explained in the Communication Process subsection.
	Next the role assignation is shown. Scrum Master: Daniel Mendoza. Product Owner: All the team will be involved. Development team: All the team will be involved.
	\paragraph{Technical Approach: Java} 
	
	\paragraph{Quality Management: Unit tests and Test Apps} How we do unit test (?). The method (?). The integration test. Frequency(?)
\subsection{Project Schedule}
 For this application, we have total 4 iterations, with 2 weeks of time for each iteration after the Intermediate Report (1st Report). 
 	\begin{table}[ht]
		\caption{Project Milestones} % title of Table
		\centering % used for cantering table
		\begin{tabular}{p{2cm}|p{4cm}|p{2cm}|p{4cm}} % centered columns
		\hline\hline %inserts double horizontal lines
		Milestone & Deliverable & Date & Dependant Upon \\ [0.5ex] % heading
		\hline
		
		1st Report & Intermediate Report complete (Iteration 0) & Feb 9 & Intermediate Planning and Initial Progress\\
		
		Initial & Application Ready (Iteration 1) & Feb 18 & First iteration of bugs fixed. Iteration 1 features implemented. Program ready for unit test.  \\ 
		
		Mid & Application Ready (Iteration 2) & Mar 3 &  Second iteration of bugs fixed. Iteration 2 features implemented. Program ready for unit test.\\
		
		Pre-Final & Application ready (Iteration 3) & Mar 17 & All features and bugs fixed. Program finalised. \\
		
		Final & Report complete and application ready (Iteration 4)  & Mar 31 & Completion of packaging and program. Final report ready. All testing done. \\[1ex] % [1ex] adds vertical space
		\hline
		\end{tabular}
		\label{table:milestones} % used to refer this table in the text
	\end{table}

\subsection{Initial Progress}


\section{Project Organisation}
\subsection{Roles}
\subsection{Collaboration Tools}
\begin{itemize}
	\item Github
	\item Trello
\end{itemize}
\subsection{Process Handling Peer Assessment}
criteria: punctual, communicated, proactive, functionally sufficient. 
\subsection{Communication Process}
\begin{itemize}
	\item Meeting: We replaced the daily scrum meetings by having two weekly meetings Mondays and Thursdays.
	\item Tools: Trello to keep track of tasks and share progress. Whatsapp group.
	\item Conflict Handling
\end{itemize}

\end{document}
