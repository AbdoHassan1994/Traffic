\documentclass[11pt]{article}


\author{KeepItClean}
\title{Traffic Simulator}

\begin{document}

\maketitle

\section{Project Description}
\subsection{Project Aim}
\begin{flushleft}
\textbf{Project Background}\par	
\end{flushleft}

Traffic simulation software plays an important role in defining the effective traffic management policies. A traffic policy must be simulated and verified before it is actually implemented in real life. Otherwise, it might harm road users’ safety and traffic efficiency.

There are a lot of factors contributing to road accidents and traffic efficiency. The major factors include speed, traffic light timing and driver’s behaviours. Therefore, this project will simulate traffic management policies focusing on those 3 main factors ie. Speed limit; Traffic light timing; Driver’s behaviours.

The software will test different policies relating to these factors and analyse the traffic efficiency and safety in relation to traffic density, enforcement policy and driver’s behaviours.

The analysis will be based on the following metrics:
\begin{itemize}
	\item Probability of crashes (\%): percentage of crashes in total vehicles per session. This metric is to measure the safety of the policy
	\item Average speed: average speed of vehicles in a session. This metric is to measure traffic efficiency and reliability.
\end{itemize}

\begin{flushleft}
\textbf{Project Scope} \par	
\end{flushleft}

The software is to simulate the traffic management policy following UK highway code which is left-lane oriented and using the speed limit within the range of speed limit defined in the UK highway code.


Within the constraints of time and resources, the project’s scope includes:
\begin{itemize}
\item Controlled Map: Minimum multiple-lane roads and a junction. It should support the scalability to a complex traffic network.
\item Vehicles: Simulate multiple types of vehicles, which include at least cars and ambulances (emergency vehicle) and at least three classes of driver’s behaviours (reckless, cautious, normal).
\item Policy: the project must support at least fixed control policy (traffic light timing, speed limit). It should be scalable to variable control policy.
\item Simulation engine: must test the policies with different levels of traffic density.
\item Report: Must provide the statistics and calculates metrics as above.
\end{itemize}


\subsection{Project Approach}
	\paragraph{Management Approach: Scrum} The team decided that due the nature of the project, we need a strategy that allowed increments of progress, as well as a product ready to deliver. We are going to follow the Scrum methodology focused in the team goals and needs. Even when people will be assigned to a specific role, it does not mean that she or he will be only do tasks related. Tasks will be assigned according to the Sprint number, task complexity, and task completion percentage. The Scrum methodology will be modified according team needs and only task progress will be reported.
	
	Taking in account the available development weeks, the Sprint length was defined to last two weeks. At the beginning of each sprint a Scrum Planning Meeting will be held. This is where the team will decide what will be the next target deliverable for the sprint. The daily stand up meeting will be replaced by mechanisms explained in the Communication Process subsection.

	
	\paragraph{Technical Approach: Java} The team made the decision to develop in Java SE because it was the technology whom all team members have at least some experience. 
	
	The next thing that motivate us to chose this technology was the necessity to reduce complexity as much as possible from the beginning due a tight development schedule. 
	
	By using Java we are removing the need of have a manual memory management in order to avoid memory leaks.
	
	Java also give us a way to easily deploy the software in many platforms without having special implementations for each one. This reduces development time and removes the platform concerns.
	
	\paragraph{Quality Management: Unit tests and Test Apps} Quality of the software will be tracked during every iteration. The strategy we are going to follow is create unit tests for each subsystem, as well as testing applications when necessary.  
	
	To develop unit test we are going to use the JUnit framework. It is a widely used way to develop unit test, therefore it will be easy to investigate implementation strategies for our project. 
	
	Each sprint will have two test tasks by default: Integration Test and Regression Test. The integration test will consist in test the subsystems interactions that changed after the development work has being completed. The regression test will consist on a set suite of predefined minimal features of the system that have to be executed in order to make sure previous functionality works as expected.
	
\subsection{Project Schedule}

 For this application, we have total 4 iterations after the Intermediate Report (1st Report). For each iteration, we use that as milestone. Detail for milestone and iterations plus the project functionality can be seen in Table \ref{table:iteration} below.


\begin{table}[p]
		\caption{Project Iteration}
		\centering
		\begin{tabular}{p{2.5cm}|p{3.5cm}|p{2.5cm}|p{2.5cm}|p{2cm}}
		\hline\hline
		Iteration0 (Feb 10) & Iteration1 (Feb 25) & Iteration2 (Mar 3) & Iteration3 (Mar 22) & Iteration4 (Mar 31)\\ [0.5ex]
		\hline
		
		Requirements & Map-Lane & Map-Roundabout & Map-Complex & Final Report\\[1ex]
		
		System Architecture Design & Map-Junction & Map-Multiple Roundabout & Vehicle-Buses & Final Bug Fixing\\ [1ex]
		
		Component Design & Map-Traffic Light & Vehicle-Driver's Behaviour & Policy-VariableControl & Final Testing\\ [1ex]
		
		Initial Report & Map-Road & Vehicle-Emergency & GUI- consolidated added vehicles/maps and animation \\[1ex]
		
		& Map-Network Boundary & GUI-Roundabout & Optimal Simulation Engine \\[1ex] 
		
		& GUI-Lane & GUI-Consolidated and Animation & \\[1ex]
		
		& GUI-Junctions Traffic Light & Data log and analysis & &\\[1ex]
		& GUI-Multiple Vehicles &&&\\[1ex]
		& Vehicle-Generic &&& \\[1ex]
		& Vehicle-Car &&& \\[1ex]
		& Vehicle-Acc Deacc &&& \\[1ex]
		& Vehicle-Traffic Light &&& \\[1ex]
		& Vehicle-Change Lane Direction &&& \\[1ex]
		& Simulation Engine &&& \\[1ex]
		& Policy-Fixed Control Policy &&&\\
		
		\hline
		\end{tabular}
		\label{table:iteration} 
		\end{table}

	
\subsection{Initial Progress}
Our works so far is the design of the application, including the UML for classes and objects that we need. 

\section{Project Organisation}

\subsection{Role Management}

The group has devised a modular multi-tiered architecture. This will aid in our our organisation as individuals within the group can work on a specific module, module class or function of the program and later integrate it within the collective design. This will also allow the group to rotate positions and collaborate for a bigger depth of understanding.


Our roles are based on 5 key components:
	\begin{itemize}
		\item Daniel will be building the session manager that overlooks the program’s operation. He is also the scrum master.
		\item Retno will design the the vehicle, or the vehicle object, and populate it with properties
		\item Rosie will design the policies
		\item Ian will work on the roads, or the environment in which the vehicles interact,
		\item Abdel-Rahman will work on the visual GUI component of the software


Since the project will be relatively of small to medium size, the group will facilitate job rotation at every iteration of development. This will be done to increase depth and breadth of understanding of the project by all members, as well as allow for a more flexible work distribution. 
Meeting procedures:
\subsection{Meeting Procedures}


The team meets regularly on Monday of each week, and arranges further meetings during the week if needed. The meeting agenda is guided by the aforementioned schedule and milestones, and is devised by the team members periodically. Real-time informal communication is done through a dedicated whatsapp group.

\subsection{Collaboration Tools}

Other than Githib for sourcefile development, the team uses Trello for taskllist managememt, Whatsapp for informal communication and googlemainling list for deliverables. Google Hangouts will be used as a video conferencing tool when nesscary.


\subsection{Process Handling Peer Assessment}
For the purpose of peer assessment the team has devised four criteria by which they will judge their own
and each other's success. Higher points will be awarded to team members who display their ability to follow the criteria. Points will also be awarded for team members who go above and beyond what could be reasonably expected of them.

	\item Punctuality/ Availability
	\subitem Showing up for meetings and joint coding sessions on time, as well as submitting required work on time.
	Making oneself available for pair/group design, coding, and testing sessions.
	
	\item Communication
	\subitem Maintaining communication with the team throughout the process. Being open to questions about your design decisions. Being a team player.
	
	\item Innovation/ Pro-activity
	\subitem Suggesting imaginative ideas, finding innovative solutions to the software decisions.
	
	\item Functionality
	\subitem Following through on the software deliverables. Making sure that you deliver the best piece of software you can by following the principles of software development, design and testing.
\end{itemize}

\subsection{Communication Process}
\begin{itemize}
	\item Meeting: We replaced the daily scrum meetings by having two weekly meetings Mondays and Thursdays.
	\item Tools: Trello to keep track of tasks and share progress. Whatsapp group.
	\item Conflict Handling
\end{itemize}

\end{document}