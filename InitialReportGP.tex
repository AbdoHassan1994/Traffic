\documentclass[11pt]{article}


\author{KeepItClean}
\title{Traffic Simulator}
\date{February 2016}


\begin{document}


\maketitle

\section{Project Description}
\subsection{Project Aim}
\begin{flushleft}
\textbf{Project Background}\par	
\end{flushleft}

Traffic simulation software plays an important role in defining the effective traffic management policies. A traffic policy must be simulated and verified before it is actually implemented in real life. Otherwise, it might harm road users’ safety and traffic efficiency.

There are a lot of factors contributing to road accidents and traffic efficiency. The major factors include speed, traffic light timing and driver’s behaviours. For instance, in 2013, in the UK, 3,064 people were killed or seriously injured in crashes where speed was a factor.

Therefore, this project will simulate traffic management policies focusing on those 3 main factors ie. Speed limit; Traffic light timing; Driver’s behaviours.

The software will test different policies relating to these factors and analyse the traffic efficiency and safety in relation to traffic density, enforcement policy and driver’s behaviours.

The analysis will be based on the following metrics:
\begin{itemize}
	\item Probability of crashes (\%): percentage of crashes in total vehicles per session. This metric is to measure the safety of the policy
	\item Average speed: average speed of vehicles in a session. This metric is to measure traffic efficiency and reliability.
\end{itemize}

\begin{flushleft}
\textbf{Project Scope} \par	
\end{flushleft}

The software is to simulate the traffic management policy following UK highway code which is left-lane oriented and using the speed limit within the range of speed limit defined in the UK highway code.


Within the constraints of time and resources, the project’s scope includes:
\begin{itemize}
\item Controlled Map: Minimum multiple-lane roads and a junction. It should support the scalability to a complex traffic network (ie. Multiple roundabouts, multiple junctions, multiple traffic lights, bus lane).
\item Vehicles: Simulate multiple types of vehicles, which include at least cars and ambulances (emergency vehicle) and at least three classes of driver’s behaviours (reckless, cautious, normal).
\item Policy: the project must support at least fixed control policy (traffic light timing, speed limit). It should be scalable to variable control policy.
\item Simulation engine: must test the policies with different levels of traffic density.
\item Report: Must provide the statistics and calculates metrics as above.
\end{itemize}


\subsection{Project Approach}
	\paragraph{Management Approach: Scrum} The team decided that due the nature of the project it was needed an strategy that allowed progress every predetermine period of time and, at the same time, to have something ready to deliver. We are going to follow the Scrum methodology focused in the team goals and needs. Even when people will be assigned to an specific role, it does not mean that she or he will be only do tasks related. Tasks will be assigned according to the Sprint number, task complexity, and task completion percentage. The Scrum methodology will be modified according team needs and only task progress will be reported.
	
	Taking in count the available development weeks, the Sprint length was defined to last two weeks. At the beginning of each sprint a Scrum Planning Meeting will be held. This is where the team will decide what will be the next target deliverable for the sprint. The daily stand up meeting will be replaced by mechanisms explained in the Communication Process subsection.
	Next the role assignation is shown. Scrum Master: Daniel Mendoza. Product Owner: All the team will be involved. Development team: All the team will be involved.
	
	\paragraph{Technical Approach: Java} The team made the decision to develop in Java SE because it was the technology whom all team members have at least some experience. 
	
	The next thing that motivate us to chose this technology was the necessity to reduce complexity as much as possible from the beginning due a tight development schedule. 
	
	By using Java we are removing the need of have a manual memory management in order to avoid memory leaks.
	
	Java also give us a way to easily deploy the software in many platforms without having special implementations for each one. This reduces development time and removes the platform concerns.
	
	\paragraph{Quality Management: Unit tests and Test Apps} Quality of the software will be tracked during every iteration. The strategy we are going to follow is create unit tests for each subsystem, as well as testing applications when necessary.  
	
	To develop unit test we are going to use the JUnit framework. It is a widely used way to develop unit test, therefore it will be easy to investigate implementation strategies for our project. 
	
	Each sprint will have two test tasks by default: Integration Test and Regression Test. The integration test will consist in test the subsystems interactions that changed after the development work has being completed. The regression test will consist on a set suite of predefined minimal features of the system that have to be executed in order to make sure previous functionality works as expected.
	
\subsection{Project Schedule}

 For this application, we have total 4 iterations after the Intermediate Report (1st Report). Each iteration we counted that as milestone. Detail for milestone and iteration can be seen in Table \ref{table:iteration} below.


\begin{table}[h]
		\caption{Project Iteration}
		\centering
		\begin{tabular}{p{2.5cm}|p{4cm}|p{2.5cm}|p{2.5cm}|p{2cm}}
		\hline\hline
		Iteration0 & Iteration1 (Feb 18) & Iteration2 (Mar 3) & Iteration3 (Mar 17) & Iteration4 (Mar 31) \\ [0.5ex]
		\hline
		
		Requirements & FT1:Map-Lane & FT5:Map-Roundabout & FT7-2:Map-Complex & Final Report\\[1ex]
		
		System Architecture Design & FT2:Map-Junction & FT7-1:Map-Multiple Roundabout & FT-10:Vehicle-Buses & Final Bug Fixing\\ [1ex]
		
		Component Detailed Design & FT3:Map-Traffic Light & FT14:Vehicle-Driver's Behaviour & FT17:Policy-VariableControl & Final Testing\\ [1ex]
		
		Initial Report & FT4:Map-Road & FT15:Vehicle-Emergency & FT22-2:GUI- Consolidated added classes \\[1ex]
		
		& FT6:Map-Network Boundary & FT20:GUI-Roundabout & FT23-2:Optimal Simulation Engine \\[1ex] 
		
		& FT18:GUI-Lane & FT21:GUI-Consolidated and Animation & \\[1ex]
		
		& FT19:GUI-Junctions Traffic Light & FT24:Data log and analysis & &\\[1ex]
		& FT21:GUI-Multiple Vehicles &&&\\
		& FT8:Vehicle-Generic &&& \\
		& FT9:Vehicle-Car &&& \\
		& FT12:Vehicle-Acc Deacc &&& \\
		& FT11:Vehicle-Traffic Light &&& \\
		& FT13:Vehicle-Change Lane Direction &&& \\
		& FT23:Simulation Engine &&& \\
		& FT16:Policy-Fixed Control Policy &&&\\
		
		
		
		\hline
		\end{tabular}
		\label{table:iteration} 
		\end{table}

	
\subsection{Initial Progress}
Our works so far is the design of the application, including the UML for classes and objects that we need. 

\section{Project Organisation}
\subsection{Roles}
Team members will be assigned to an specific role, however this does not mean that she or he will be only doing tasks related. This ones will be balanced and assigned according to the Sprint goal, task complexity, and task completion percentage.
	\begin{itemize}
		\item Scrum Master: Daniel Mendoza. 
		\item Product Owner: All the team will be involved.
		\item Development team: All the team will be involved.
	\end{itemize}
\subsection{Collaboration Tools}
\begin{itemize}
	\item Github
	\item Trello
\end{itemize}
\subsection{Process Handling Peer Assessment}
criteria: punctual, communicated, proactive, functionally sufficient. 
\subsection{Communication Process}
\begin{itemize}
	\item Meeting: We replaced the daily scrum meetings by having two weekly meetings Mondays and Thursdays.
	\item Tools: Trello to keep track of tasks and share progress. Whatsapp group.
	\item Conflict Handling
\end{itemize}

\end{document}
